%------------------------------------------------------------------------------
% Document setup
%------------------------------------------------------------------------------
\documentclass[american, twoside]{article}
%-------------------------------------------------------------------------
% Packages
%-------------------------------------------------------------------------
% Ability to reference the number of pages in document
\usepackage{lastpage}
% Nice equations
\usepackage{mathtools}
% Skip a line between paragraphs
\usepackage[skip=10pt plus1pt, indent=40pt]{parskip}
% Alter the margins, twoside tells it to alternate left/right margins on even/odd pages for
% two-sided printing
\usepackage[twoside, letterpaper, lmargin=0.75in, rmargin=1.25in]{geometry}
% Fancy Headers/Footers
\usepackage{fancyhdr}
% Adjust table of contents formatting
\usepackage{tocloft}
% Provide hypertext links
\usepackage{hyperref}
% Custom datetime formatting
\usepackage{datetime2}
% Language localization
\usepackage[american]{babel}
% Boxes around stuff (Abstract and ToC)
\usepackage{tcolorbox}
% Make some micro-typographical corrections to improve appearance
\usepackage{microtype}
% Reference management
\usepackage[sorting=nty, backref=true, backend=biber, hyperref=true, autocite=superscript, sortcites=true]{biblatex}
%-------------------------------------------------------------------------
% End Packages
%-------------------------------------------------------------------------

%-------------------------------------------------------------------------
% Custom date format
%-------------------------------------------------------------------------
\DTMnewdatestyle{strunkdate}{%
  \renewcommand*{\DTMdisplaydate}[4]{\number##3\ \DTMenglishmonthname{##2} ##1}
  \renewcommand*{\DTMDisplaydate}{\DTMdisplaydate}}
\AtBeginDocument{\DTMsetdatestyle{strunkdate}}
%-------------------------------------------------------------------------
% End Custom date format
%-------------------------------------------------------------------------

%-------------------------------------------------------------------------
% Article meta-data
%-------------------------------------------------------------------------
\title{\Huge Filtering Seismic Timeseries Data in Passive-source Seismic-Processing (PsSp)\\
\large Implementation details of the filters provided for isolating seismic signals}
\author{Alexander R. Blanchette\thanks{arbCoding@gmail.com}}
\date{\Today}
\addbibresource{filters.bib}
%-------------------------------------------------------------------------
% End Article meta-data
%-------------------------------------------------------------------------

%-------------------------------------------------------------------------
% Draft formatting options
%-------------------------------------------------------------------------
% These can be commented out to get a finalized version
% Place line numbers in the left-side margin
\usepackage[left]{lineno}
\linenumbers
% Double-spacing
\usepackage{setspace}
\doublespacing
%-------------------------------------------------------------------------
% End Draft formatting options
%-------------------------------------------------------------------------

%-------------------------------------------------------------------------
% Miscellaneous Formatting
%-------------------------------------------------------------------------
\fancyhf{} % clear existing header/footer entries
% Place Page X of Y on outer-edge of footer
\fancyfoot[RO, LE]{Page \thepage \hspace{1pt} of \pageref{LastPage}}
\fancyhead{} % clear all header fields
\fancyhead[RO, LE]{\textbf{Filtering Seismic Timeseries Data}}
\fancyhead[C]{Blanchette}
\fancyhead[LO, RE]{2023}
% Not so ugly anymore!
\hypersetup{colorlinks=true, linkcolor=black, citecolor=blue, filecolor=magenta, urlcolor=blue, pdftitle={Filtering Seismic Timeseries Data}, pdfpagemode=FullScreen}
\urlstyle{same}
%-------------------------------------------------------------------------
% End Miscellaneous Formatting
%-------------------------------------------------------------------------

%------------------------------------------------------------------------------
% End Document setup
%------------------------------------------------------------------------------

%------------------------------------------------------------------------------
% Article Content
%------------------------------------------------------------------------------
\begin{document}
\maketitle
\pagestyle{fancy}
%-------------------------------------------------------------------------
% Abstract
%-------------------------------------------------------------------------
\begin{tcolorbox}
    \section*{Abstract} \label{Abstract}
    Seismic data---whether it is ground motion (displacement), velocity, or acceleration---is recorded as a function of time (timeseries). Seismic timeseries
    recordings contain information from a plethora of sources---natural and anthropogenic. Typically, an analysis task is focused on processing/analyzing
    signals from a specific target \textit{source}\footnote{\textit{Source} is being used in the generic sense here; it could refer to the nucleation point of the seismic
    signals (earthquakes, explosions, etc.) or to structural sources (reflections, refractions, conversions).}. Generally, seismometers record and are sensitive
    to a frequency range that is significantly larger than that which any given analytical method/goal requires. Filtering is a critically important tool for
    isolating signals of interest. In this document I give a brief overview of timeseries filtering in general and provide the specific details of the filters
    implemented in Passive-source Seismic-processing (PsSp).
\end{tcolorbox}
%-------------------------------------------------------------------------
% End Abstract
%-------------------------------------------------------------------------

%-------------------------------------------------------------------------
% ToC
%-------------------------------------------------------------------------
\newpage
\begin{tcolorbox}
    %\renewcommand*\contentsname{Organization}
    \renewcommand\cftsecfont{\large}
    \renewcommand\cftsecpagefont{\small}
    \tableofcontents
\end{tcolorbox}
%-------------------------------------------------------------------------
% End ToC
%-------------------------------------------------------------------------

%-------------------------------------------------------------------------
% Introduction
%-------------------------------------------------------------------------
\section{Introduction} \label{Introduction}
Filtering of time-dependent recordings (timeseries data) is a task common among a wide range of different scientific/engineering disciplines.
While great advances have been made to analyze as much of the seismic timeseries as possible---such as the extraction of approximate
Green's functions from the ambient seismic noise field---virtually all seismic analysis first requires the seismic timeseries to
be filtered in order to isolate the portion of the recorded data that is most compatible with the intended analysis. In the fairly
distant past this was less important due to the limitations of: 1) the available seismographic instruments, and 2) the available
computational infrastructure. At present, relatively inexpensive seismometers can record broadband seismic timeseries with the flat-sensitivity
range of a high-quality instrument being between a period of 1000 seconds (s) at the low end (a frequency of 0.001 Hz) up to a high
of 100 Hz. Such a large dynamic range of frequencies---containing signals from numerous stationary and non-stationary sources---results in a
rather complicated timeseries. This excess information convolutes the analysis of the signal(s) of interest.
Filtering is the solution to this problem; it allows the analyst to isolate the signal(s) of interest by removing the \textit{noise}\footnote{\textit{Noise} is a bit of a misunderstood term in
seismology, especially with the advances in the last few decades in ambient noise seismology. The word 'noise' is used here to mean 'any
signal that is not of interest for a specific analysis' and does not provide any suggestion to the source of the noise.} from the
recorded timeseries. Fortunately---thanks to the many technological advancements over the last few decades---personal computers are sufficiently
capable of handling most modern seismic workflows (all except particularly large or high throughput workflows that require more specialized
computing infrastructure).

Filtering is a tremendously wide topic of study itself, with a rich history across many otherwise disparate fields of study, well
beyond the scope of this simple document. Therefore, I will restrict myself to---at most---provide a brief discussion
on the fundamentals of filtering, while leaving many of the details up for the intrepid reader to research on their own. Afterward,
I will provide a detailed description of the implementation of filters specific to \href{https://github.com/arbCoding/PsSp}{PsSp}.
This is not meant to provide a detailed derivation from first principles of the filters employed, but is instead intended to serve as
a pragmatic guide on their internal implementation. This is intended as much as a guide to any reader as it is to myself.
%-------------------------------------------------------------------------
% End Introduction
%-------------------------------------------------------------------------

%-------------------------------------------------------------------------
% Background
%-------------------------------------------------------------------------
\section{Background} \label{Background}
%--------------------------------------------------------------------
% Historical Background
%--------------------------------------------------------------------
\subsection{Historical Background} \label{Background:History}
%--------------------------------------------------------------------
% End Historical Background
%--------------------------------------------------------------------

%--------------------------------------------------------------------
% Mathematical Background
%--------------------------------------------------------------------
\subsection{Mathematical Background} \label{Background:Math}

\subsubsection{The Temporal (Time) Domain} \label{Background:Math:Temporal}

\subsubsection{The Spectral (Frequency) Domain} \label{Background:Math:Spectral}

\subsubsection{The Laplace (Complex-Frequency) Domain} \label{Background:Math:Laplace}
%--------------------------------------------------------------------
% End Mathematical Background
%--------------------------------------------------------------------
%-------------------------------------------------------------------------
% End Background
%-------------------------------------------------------------------------

%-------------------------------------------------------------------------
% Usage of FFTW
%-------------------------------------------------------------------------
\section{The Fast Fourier Transform} \label{FFT}
%-------------------------------------------------------------------------
% End Usage of FFTW
%-------------------------------------------------------------------------

%-------------------------------------------------------------------------
% The Butterworth Filter
%-------------------------------------------------------------------------
\section{The Butterworth Filter} \label{Butterworth}
Testing\autocite{butterworthWikipedia}
\subsection{Lowpass} \label{Butterworth:Lowpass}
\begin{equation} \label{butterworth:lowpass_generic}
    H(s) = \frac{G_{0}}{B_{n}(a)}; a = \frac{s}{w_{c}}
\end{equation}

\subsection{Highpass} \label{Butterworth:Highpass}
\begin{equation} \label{butterworth:polynomial}
    B_{n}(s) = \sum_{k = 0}^{n} a_{k}s^{k}
\end{equation}

\subsection{Bandpass} \label{Butterworth:Bandpass}
\subsection{Bandreject} \label{Butterworth:Bandreject}
\begin{equation} \label{butterowrth:coefficients}
    a_{k + 1} = a_{k}\frac{cos\left(k\gamma\right)}{sin((k + 1)\gamma)}; a_{0} = 1; \gamma = \frac{\pi}{2n}; a_{k} = a_{n - k}
\end{equation}
%-------------------------------------------------------------------------
% End The Butterworth Filter
%-------------------------------------------------------------------------

%-------------------------------------------------------------------------
% References
%-------------------------------------------------------------------------
% Always want the references to start on a new page
\newpage
\printbibliography[heading=bibintoc]
%-------------------------------------------------------------------------
% End References
%-------------------------------------------------------------------------

\end{document}
%------------------------------------------------------------------------------
% End Article Content
%------------------------------------------------------------------------------