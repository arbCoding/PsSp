%------------------------------------------------------------------------------
% Document setup
%------------------------------------------------------------------------------
\documentclass[12pt, letterpaper]{article}
%-------------------------------------------------------------------------
% Packages
%-------------------------------------------------------------------------
% Nice equations
\usepackage{mathtools}
% Provide hypertext links
\usepackage{hyperref}
% Skip a line between paragraphs
\usepackage[skip=10pt plus1pt, indent=40pt]{parskip}
% Reduce the margins
\usepackage[margin=0.5in]{geometry}
%-------------------------------------------------------------------------
% End Packages
%-------------------------------------------------------------------------

%-------------------------------------------------------------------------
% Article meta-data
%-------------------------------------------------------------------------
\title{The Implementation of Seismic Filters in Passive-source Seismic Processing}
\author{Alexander R. Blanchette\thanks{arbCoding@gmail.com}}
\date{3 July\\2023}
%-------------------------------------------------------------------------
% End Article meta-data
%-------------------------------------------------------------------------

%------------------------------------------------------------------------------
% End Document setup
%------------------------------------------------------------------------------

%------------------------------------------------------------------------------
% Article Content
%------------------------------------------------------------------------------
\begin{document}
\maketitle
\noindent\rule{\textwidth}{1pt}

\section{Abstract} \label{Abstract}

%\section{Purpose} \label{Purpose}
%The purpose of this document is to provide information on the filters provided by

\section{Introduction} \label{Introduction}
Filtering of time-varying signals is a common task across a wide-swath of different scientific/engineering disciplines.
While great advances have been made to analyze as much of the seismic signal as possible, even the extraction of approximate
Green's functions from the ambient seismic noise field, virtually all seismic analysis first requires the seismic signal to
be filtered in order to isolate the portion of the signal that is most of interest to the analyst's plans. In the fairly
distant past this was less important due to the limitations of the instruments available. At present, relatively inexpensive
seismometers can record broadband seismic signals with the typical flat-reponse range of a high-quality instrument being between
a period of 1000 seconds (s) at the low end (a frequency of 0.001 Hz) up to a high of 100 Hz. With such a large dynamic range
of frequencies over which the signal is recorded much overlapping information is recorded. This excess information convolutes
the analysis of the signals of interest. Filtering is the solution to this problem, filtering allows the analyst to isolate
the signal of interest by removing the noise\footnote{'Noise' is a bit of a misunderstood term in seismology, especially with the advances in
the last few decades in ambient noise seismology. I use the word 'noise' to mean 'any signal that is not of interest for a specific
analysis' and not to provide any information of the source of the noise.}.

Filtering is a tremendously wide topic of study itself, with a rich history across many otherwise disparate fields of study, well
beyond the scope of this simple documentation. Therefore, I will restrict mysellf in this document to provide a brief discussion
on the fundamentals of filtering, while leaving many of the details up for the intrepid reader to research on their own. Afterward
I will provide a detailed description of the implementation of filters specific to \href{https://github.com/arbCoding/PsSp}{PsSp}.
This is not meant to provide a detailed derivation of the filters employed from first principles, but is instead intended to serve as
a pragmatic guide on their internal implementation. This is intended as much as a guide to any reader as it is to myself.

\begin{equation} \label{butterworth:lowpass_generic}
    H(s) = \frac{G_{0}}{B_{n}(a)}; a = \frac{s}{w_{c}}
\end{equation}

\begin{equation} \label{butterworth:polynomial}
    B_{n}(s) = \sum_{k = 0}^{n} a_{k}s^{k}
\end{equation}

\begin{equation} \label{butterowrth:coefficients}
    a_{k + 1} = a_{k}\frac{cos\left(k\gamma\right)}{sin((k + 1)\gamma)}; a_{0} = 1; \gamma = \frac{\pi}{2n}; a_{k} = a_{n - k}
\end{equation}

\end{document}
%------------------------------------------------------------------------------
% End Article Content
%------------------------------------------------------------------------------