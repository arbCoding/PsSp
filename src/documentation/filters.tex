%------------------------------------------------------------------------------
% Document setup
%------------------------------------------------------------------------------
\documentclass[twoside]{article}
%-------------------------------------------------------------------------
% Packages
%-------------------------------------------------------------------------
% 
\usepackage{lastpage}
% Nice equations
\usepackage{mathtools}
% Provide hypertext links
\usepackage{hyperref}
% Skip a line between paragraphs
\usepackage[skip=10pt plus1pt, indent=40pt]{parskip}
% Alter the margins
\usepackage[twoside, letterpaper, lmargin=0.75in, rmargin=1.25in]{geometry}
% Fancy Headers/Footers
\usepackage{fancyhdr}
%-------------------------------------------------------------------------
% End Packages
%-------------------------------------------------------------------------

%-------------------------------------------------------------------------
% Article meta-data
%-------------------------------------------------------------------------
\title{\Huge Filtering Seismic Timeseries Data in Passive-source Seismic-Processing (PsSp)\\
\large Implementation details of the filters provided for isolating seismic signals}
\author{Alexander R. Blanchette\thanks{arbCoding@gmail.com}}
\date{3 July\\2023}
%-------------------------------------------------------------------------
% End Article meta-data
%-------------------------------------------------------------------------

%-------------------------------------------------------------------------
% Miscellaneous Formatting
%-------------------------------------------------------------------------
\fancyhf{} % clear existing header/footer entries
% Place Page X of Y on the right-hand
% side of the footer
\fancyfoot[R]{Page \thepage \hspace{1pt} of \pageref{LastPage}}
\fancyhead{} % clear all header fields
\fancyhead[RO,LE]{\textbf{Filtering Seismic Timeseries Data}}
\fancyhead[C]{Blanchette}
\fancyhead[LO,RE]{2023}
%-------------------------------------------------------------------------
% End Miscellaneous Formatting
%-------------------------------------------------------------------------

%------------------------------------------------------------------------------
% End Document setup
%------------------------------------------------------------------------------

%------------------------------------------------------------------------------
% Article Content
%------------------------------------------------------------------------------
\begin{document}
\maketitle
%\noindent\rule{\textwidth}{1pt}
\pagestyle{fancy}
%-------------------------------------------------------------------------
% Abstract
%-------------------------------------------------------------------------
\section{Abstract} \label{Abstract}
Seismic data---whether it is ground motion (displacement), velocity, or acceleration---is recorded as a function of time. Seismic timeseries
recordings contain information from a plethora of sources---natural or anthropogenic. Typically, an analysis task is focused on processing/analyzing
signals from a specific target 'source'\footnote{'Source' is being used in the generic sense here; it could refer to the nucleation point of the seismic
signals (earthquake, explosion, etc.) or to structural sources (reflections, refractions, conversions)}. Generally, seismometers record and are sensitive
to a frequency range that is significantly larger than that which any given analyitcal method/goal requires. Filtering is a critically important tool for
isolating signals of interest. In this document I give a brief overview of timeseries filtering in general and provide the specific details of the filters
implemented in Passive-source Seismic-processing (PsSp).
%-------------------------------------------------------------------------
% End Abstract
%-------------------------------------------------------------------------

%-------------------------------------------------------------------------
% Introduction
%-------------------------------------------------------------------------
\section{Introduction} \label{Introduction}
Filtering of time-dependent recordings (timeseries data) is a task common among a wide range of different scientific/engineering disciplines.
While great advances have been made to analyze as much of the seismic timeseries as possible, even the extraction of approximate
Green's functions from the ambient seismic noise field, virtually all seismic analysis first requires the seismic timeseries to
be filtered in order to isolate the portion of the recorded data that is most compatible with the analysis planned. In the fairly
distant past this was less important due to the limitations of: 1) the available seismographic instruments, and 2) the available
computational infrastructure. At present, relatively inexpensive seismometers can record broadband seismic timeseries with the flat-sensitivity
range of a high-quality instrument being between a period of 1000 seconds (s) at the low end (a frequency of 0.001 Hz) up to a high
of 100 Hz. Such a large dynamic range of frequencies---containing signals from numerous stationary and non-stationary sources---results in a
rather complicated timeseries. This excess information convolutes the analysis of the signal(s) of interest.
Filtering is the solution to this problem; it allows the analyst to isolate the signal(s) of interest by removing the noise\footnote{'Noise' is a bit of a misunderstood term in
seismology, especially with the advances in the last few decades in ambient noise seismology. I use the word 'noise' to mean 'any
signal that is not of interest for a specific analysis' and not to provide any information of the source of the noise.} from the
recorded timeserie. Fortunately---thanks to the many technological advancements over the last few decades---personal computers are sufficiently
capable of handling most modern seismic workflows (all except particularly large/high through-put workflows that require more specialized
computing infrastructure).

Filtering is a tremendously wide topic of study itself, with a rich history across many otherwise disparate fields of study, well
beyond the scope of this simple documentation. Therefore, I will restrict mysellf in this document to provide a brief discussion
on the fundamentals of filtering, while leaving many of the details up for the intrepid reader to research on their own. Afterward,
I will provide a detailed description of the implementation of filters specific to \href{https://github.com/arbCoding/PsSp}{PsSp}.
This is not meant to provide a detailed derivation of the filters employed from first principles, but is instead intended to serve as
a pragmatic guide on their internal implementation. This is intended as much as a guide to any reader as it is to myself.
%-------------------------------------------------------------------------
% End Introduction
%-------------------------------------------------------------------------

%-------------------------------------------------------------------------
% Background
%-------------------------------------------------------------------------
\section{Background} \label{Background}
%--------------------------------------------------------------------
% Historical Background
%--------------------------------------------------------------------
\subsection{Historical Background} \label{Background:History}
%--------------------------------------------------------------------
% End Historical Background
%--------------------------------------------------------------------

%--------------------------------------------------------------------
% Mathematical Background
%--------------------------------------------------------------------
\subsection{Mathematical Background} \label{Background:Math}

\subsubsection{The Temporal (Time) Domain} \label{Background:Math:Temporal}

\subsubsection{The Spectral (Frequency) Domain} \label{Background:Math:Spectral}

\subsubsection{The Laplace (Complex-Frequency) Domain} \label{Background:Math:Laplace}
%--------------------------------------------------------------------
% End Mathematical Background
%--------------------------------------------------------------------
%-------------------------------------------------------------------------
% End Background
%-------------------------------------------------------------------------

%-------------------------------------------------------------------------
% Usage of FFTW
%-------------------------------------------------------------------------
\section{The Fast Fourier Transform} \label{FFT}
%-------------------------------------------------------------------------
% End Usage of FFTW
%-------------------------------------------------------------------------

%-------------------------------------------------------------------------
% The Butterworth Filter
%-------------------------------------------------------------------------
\section{The Butterworth Filter} \label{Butterworth}
\subsection{Lowpass} \label{Butterworth:Lowpass}
\begin{equation} \label{butterworth:lowpass_generic}
    H(s) = \frac{G_{0}}{B_{n}(a)}; a = \frac{s}{w_{c}}
\end{equation}

\subsection{Highpass} \label{Butterworth:Highpass}
\begin{equation} \label{butterworth:polynomial}
    B_{n}(s) = \sum_{k = 0}^{n} a_{k}s^{k}
\end{equation}

\subsection{Bandpass} \label{Butterworth:Bandpass}
\subsection{Bandreject} \label{Butterworth:Bandreject}
\begin{equation} \label{butterowrth:coefficients}
    a_{k + 1} = a_{k}\frac{cos\left(k\gamma\right)}{sin((k + 1)\gamma)}; a_{0} = 1; \gamma = \frac{\pi}{2n}; a_{k} = a_{n - k}
\end{equation}
%-------------------------------------------------------------------------
% End The Butterworth Filter
%-------------------------------------------------------------------------
\end{document}
%------------------------------------------------------------------------------
% End Article Content
%------------------------------------------------------------------------------